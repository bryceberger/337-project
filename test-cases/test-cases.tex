\documentclass{scrartcl}

\usepackage[hidelinks]{hyperref}
\usepackage{longtable, booktabs}
\usepackage{enumitem}
\usepackage{tikz}

\title{Test Planning for Cooperative Design Lab}
\subtitle{AHB-lite / USB}
\author{ Bryce Berger \and Trevor Moorman \and Robert Sammelson }
\date{November 28, 2022}

\begin{document}
\maketitle
\tableofcontents
\newpage

%\section{Top-level}

\section{AHB-lite}

\subsection{Isolated single read/write}

For every isolated test case, wait more than 1 clock cycle between read and write.

\subsubsection{4 byte read/write}
\label{s4brr}
In the following test cases, expect to read the same value as was just written. Because there is no section of memory that is both R/W and 4 bytes, we will test the addresses 0xC --- 0xF. 4 byte reads and writes should be accepted at those addresses, but addresses 0xE and 0xF are unused.
\begin{enumerate}
    \item Aligned write to 4-byte address, followed by read from same address (0xC and 0xC).
    \item Aligned write, followed by unaligned read (0xC and 0xE).
    \item Unaligned write, aligned read.
    \item Unaligned write, unaligned read.
\end{enumerate}

\subsubsection{2 byte read/write}
\label{s2brr}
Similarly, expect to read the same value as was just written. The memory section from 0xC to 0xD is used again, as the only (non-special) R/W memory section.
\begin{enumerate}
    \item Aligned write to 2-byte address, followed by read from same address (0xC and 0xC).
    \item Aligned write, followed by unaligned read (0xC and 0xD).
    \item Unaligned write, aligned read.
    \item Unaligned write, unaligned read.
\end{enumerate}

\subsubsection{1 byte read/write}
\begin{enumerate}
    \item
    \begin{enumerate}
        \item Write to 0xC
        \item Write to 0xD
        \item Read from 0xD
        \item Read from 0xC
    \end{enumerate}
\end{enumerate}

\subsection{Overlapping read/write}
For overlapping test cases, execute read and write on directly adjacent clock cycles.

\subsubsection{4 byte read/write}
See \autoref{s4brr} for notes about 4 byte accesses.
\begin{enumerate}
    \item Aligned write to 4-byte address, followed by read from same address (0xC and 0xC).
    \item Aligned write, followed by unaligned read (0xC and 0xE).
    \item Unaligned write, aligned read.
    \item Unaligned write, unaligned read.
\end{enumerate}

\subsubsection{2 byte read/write}
See \autoref{s2brr} for notes about 2 byte accesses.
\begin{enumerate}
    \item Aligned write to 2-byte address, followed by read from same address (0xC and 0xC).
    \item Aligned write, followed by unaligned read (0xC and 0xD).
    \item Unaligned write, aligned read.
    \item Unaligned write, unaligned read.
\end{enumerate}

\subsubsection{1 byte read/write}
\begin{enumerate}
    \item
    \begin{enumerate}
        \item Write to 0xC
        \item Write to 0xD
        \item Read from 0xD
        \item Read from 0xC
    \end{enumerate}
\end{enumerate}

\subsection{Writing to read only}
For each of these tests, \texttt{hready} and \texttt{hresp} should be as shown in \autoref{fig:errorwave}.
\begin{figure}
    \begin{center}
        \begin{tikzpicture}
            \draw (0, 2.5) node [left] {\texttt{clk}} (0, 2) -- (0, 3) -- (2, 3) -- (2, 2) -- (4, 2) -- (4, 3) -- (6, 3) -- (6, 2) -- (8, 2) -- (8, 3);
            \draw (0, 0.5) node [left] {\texttt{hready}} (0, 1) -- (1, 1) -- (1, 0) -- (5, 0) -- (5, 1) -- (8, 1);
            \draw (0, -1.5) node [left] {\texttt{hresp}} (0, -2) -- (1, -2) -- (1, -1) -- (8, -1) -- (8, -2);
        \end{tikzpicture}
    \end{center}
    \caption{Expected \texttt{hready} and \texttt{hresp} after error}
    \label{fig:errorwave}
\end{figure}
That is, \texttt{hready} extends the transaction for a clock cycle, while \texttt{hresp} asserts that there was an error with the transaction.

\begin{enumerate}
    \item 4 byte write to read only (0x6)
    \item 2 byte write to read only (0x5)
    \item 1 byte write to read only (0x8)
\end{enumerate}

\subsection{Access to unused addresses}

\subsubsection{Direct access}
If directly and exclusively written to, unused addresses should trigger an error. This will have the same response as shown in \autoref{fig:errorwave}. If read from, they should always return 0.

\begin{enumerate}
    \item 2 byte write to 0xE.
    \item 1 byte write to 0x9.
\end{enumerate}

\subsubsection{Indirect access}
There should not be an error when writing to addresses adjacent to unused space. However, regardless of what is written, these addresses should report 0 when read.

\begin{enumerate}
    \item
    \begin{enumerate}
        \item 2 byte write to 0x8.
        \item 1 byte read from 0x9.
    \end{enumerate}
    \item
    \begin{enumerate}
        \item 4 byte write to 0xC.
        \item 2 byte read from 0xF.
    \end{enumerate}
\end{enumerate}

\section{FIFO}

\subsection{Filling}
For all filling test cases, start from a clean reset. 

\subsubsection{Filling from USB RX}
\begin{enumerate}
    \item Write single byte using USB RX signals. Expect RX Data output signal to change.
    \item Write 10-20 bytes using USB RX signals. Expect RX Data output to show the first transmitted.
    \item Write 64 bytes using USB RX signals. Expect RX Data output to show the first transmitted.
\end{enumerate}

\subsubsection{Filling from AHB-lite slave}
\begin{enumerate}
    \item Write single byte using AHB-lite slave signals. Expect RX Data output signal to change.
    \item Write 10-20 bytes using AHB-lite slave signals. Expect RX Data output to show the first transmitted.
    \item Write 64 bytes using AHB-lite slave signals. Expect RX Data output to show the first transmitted.
\end{enumerate}

\subsection{Draining}
For all draining test cases, start by filling with either USB RX or AHB-lite slave signals.

\subsubsection{Draining from USB TX}
\begin{enumerate}
    \item Fill a single byte using AHB-lite. Drain with USB TX. Expect to receive byte that was input.
    \item Fill 10-20 bytes using AHB-lite. Drain with USB TX. Expect to receive bytes in order of input (first in, first out).
    \item Fill 64 bytes using AHB-lite. Drain with USB TX. Expect to receive bytes in order of input.
\end{enumerate}

\subsubsection{Draining from AHB-lite slave}
\begin{enumerate}
    \item Fill a single byte using USB RX. Drain with AHB-lite slave. Expect to receive byte that was input.
    \item Fill 10-20 bytes using USB RX. Drain with AHB-lite slave. Expect to receive bytes in order of input (first in, first out).
    \item Fill 64 bytes using USB RX. Drain with AHB-lite slave. Expect to receive bytes in order of input.
\end{enumerate}

\section{USB RX}
\subsection{Reset Verification}
The RX module should be reset, and the correctness of the default outputs should be verified.

\subsection{Nominal Packet Reception}
\subsubsection{Token Packets}
\begin{enumerate}
    \item Ignore when endpoint or address does not match
    \item Start reading data when endpoint and address match
\end{enumerate}

\subsubsection{Data Packets}
Verify reciept for at least packets of length 0, 1, and 64 bytes.

\subsubsection{Handshake Packets}
Verify reciept of ACK and NAK; verify no error on STALL.

\subsection{Erroneous Packet Reception}
The following errors should be sent to the reciever, and the reciever should assert an error. Afterwards a valid
packet should be sent, and this packet should be recieved correctly.
\begin{enumerate}
    \item Invalid sync byte
    \item Invalid PID code
    \item Incorrect negation of PID code
    \item Invalid bit stuffing
    \item Incorrect CRC (token packet)
    \item Incorrect CRC (data packet: 0 byte, 1 byte, 64 byte)
    \item Early EOP
    \item Late EOP
    \item EOP at invalid time
\end{enumerate}

\section{USB TX}
	\subsection{Correct behavior during nominal packet transmission} \label{sec:packet-trans}
		\subsubsection{ACK}
			\begin{enumerate}
            	\item Send ACK.
                \item Check 'sync' byte was properly sent.
                \item Check ACK 'pid' byte was properly sent.
                \item Check 'EOP' was properly asserted for 2 bit periods.
            \end{enumerate}
		\subsubsection{NAK}
			\begin{enumerate}
            	\item Send NAK.
                \item Check 'sync' byte was properly sent.
                \item Check NAK 'pid' byte was properly sent.
                \item Check 'EOP' was properly asserted for 2 bit periods.
            \end{enumerate}
		\subsubsection{DATA0}
        	Each DATA0 packet will include the following checks.
            \begin{enumerate}
            	\item Check 'sync' byte was properly sent.
                \item Check ACK 'pid' byte was properly sent.
                \item Check 'EOP' was properly sent.
            \end{enumerate}
            Will send DATA0 packets with the following amounts of data.
			\begin{enumerate}
            	\item 0 bytes of data.
                \item 1 byte of data.
                \item 32 bytes of data.
                \item 64 bytes of data.
            \end{enumerate}
		\subsubsection{STALL}
			\begin{enumerate}
            	\item Send STALL.
                \item Check 'sync' byte was properly sent.
                \item Check STALL 'pid' byte was properly sent.
                \item Check 'EOP' was properly asserted for 2 bit periods.
            \end{enumerate}
	
	\subsection{Correct behavior during the NRZI encoding of the outgoing bit-stream}
		Throughout all tests described in \autoref{sec:packet-trans}, ensure that both d\textsubscript{plus} and d\textsubscript{minus} properly decode to d\textsubscript{orig}.

	\subsection{Correct behavior during the bit-stuffing of the outgoing bit-stream}
		\subsubsection{Bit-stuffing at beginning of data0 packet}
        	\begin{enumerate}
            	\item Send a DATA0 packet containing one byte of data that is 0xFF.
                \item Check that a bit-stuff occurred after the fourth bit of data.
            \end{enumerate}
		\subsubsection{Bit-stuffing in middle of data}
        	\begin{enumerate}
            	\item Send a DATA1 packet containing one byte of data that is 0xFF.
                \item Check that a bit-stuff occurred after the sixth bit of data.
            \end{enumerate}
		\subsubsection{Bit-stuffing at end of data}
			\begin{enumerate}
            	\item Send a DATA1 packet containing one byte of data that is 0x3F.
                \item Check that a bit-stuff occurred after the eighth bit of data.
            \end{enumerate}
	
    \subsection{Correct inclusion of calculated CRC values}
		Throughout all tests described in \autoref{sec:packet-trans}, ensure all packets include the correct CRC value which were manually calculated beforehand.

\end{document}
